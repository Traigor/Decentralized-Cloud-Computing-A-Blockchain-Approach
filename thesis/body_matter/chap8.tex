\chapter{Επίλογος}

\section{Συμπεράσματα}
Στην παρούσα εργασία ερευνήθηκε ο χώρος των αποκεντρωμένων εφαρμογών, εστιάζοντας ειδικά σε μια \en{dApp} σχεδιασμένη για υπολογιστικό νέφος. Η \en{dApp} αυτή αντιπροσωπεύει μια σημαντική αλλαγή στον τομέα του υπολογιστικού νέφους, διαφοροποιημένο από τα παραδοσιακά συγκεντρωτικά μοντέλα, εισάγοντας ένα εκδημοκρατισμένο, διαφανές και ασφαλές οικοσύστημα. Αξιοποιώντας έναν απλό μηχανισμό επαλήθευσης, το \en{docker containerization} και ένα σύστημα κατανομής εργασιών με βάση τη δημοπρασία, η \en{dApp} αντιμετωπίζει βασικές προκλήσεις στον τομέα του υπολογιστικού νέφους. Η δυνατότητά του να αναδιαμορφώσει το τοπίο, καθιστώντας το \en{cloud computing} πιο προσιτό, οικονομικά αποδοτικό και αξιόπιστο, υπογραμμίζει τη σημασία του στην εξελισσόμενη ψηφιακή εποχή.

\section{Μελλοντικές Επεκτάσεις}
Το τοπίο του αποκεντρωμένου υπολογιστικού νέφους είναι γεμάτο δυνατότητες και ευκαιρίες ανάπτυξης. Μελλοντικά, διάφοροι τομείς αναδεικνύονται ως κομβικοί για την εξέλιξη και την ενίσχυση αυτού του τομέα:

\begin{itemize}
\item Έλεγχος κλιμακωσιμότητας: Με την αυξανόμενη υιοθέτηση αποκεντρωμένων συστημάτων, η αντιμετώπιση της επεκτασιμότητας του \en{blockchain} καθίσταται ζωτικής σημασίας. Οι καινοτομίες σε λύσεις επιπέδου (\en{layer}) 2 του \en{Ethereum blockchain} \cite{ref24} ή η διερεύνηση εναλλακτικών αλγορίθμων συναίνεσης θα μπορούσαν να είναι καθοριστικές.
\item Βελτιωμένες διεπαφές χρήστη: Για να προωθηθεί η ευρύτερη υιοθέτηση, η εμπειρία του χρήστη των αποκεντρωμένων εφαρμογών πρέπει να είναι διαισθητική και φιλική προς τον αυτόν. Μια καλύτερη και πιο αποδοτική υλοποίηση της γραφικής διεπαφής της εφαρμογής κρίνεται απαραίτητη για την βελτίωση της εμπειρίας χρήσης της.
\item Προηγμένα πρωτόκολλα ασφαλείας: Τα μέτρα ασφαλείας του \en{dApp} θέτουν ισχυρά θεμέλια, αλλά η δυναμική φύση των απειλών στον κυβερνοχώρο σημαίνει ότι τα πρωτόκολλα ασφαλείας πρέπει να εξελίσσονται συνεχώς. Αυτό θα μπορούσε να περιλαμβάνει την ενσωμάτωση προηγμένων κρυπτογραφικών τεχνικών κατά την εκτέλεση της υπολογιστικής εργασίας ή την λήψη των αποτελεσμάτων των υπολογισμών της από τον πελάτη. Επιπλέον, η παρακολούθηση των εξελίξεων και εφαρμογή των σύγχρονων μέτρων ασφαλείας για το \en{docker containerization} κρίνεται απαραίτητη \cite{ref46,ref47}.
\item Διαλειτουργικότητα: Καθώς το αποκεντρωμένο οικοσύστημα επεκτείνεται, η εξασφάλιση απρόσκοπτων αλληλεπιδράσεων μεταξύ διαφορετικών πλατφορμών και συστημάτων \en{blockchain} θα είναι σημαντική. Αυτό απαιτεί την ανάπτυξη λύσεων \en{cross-chain} και τυποποιημένων πρωτοκόλλων.
\item Έρευνα για οικονομικά αποδοτικές τεχνολογίες \en{blockchain}: Ένας από τους αποτρεπτικούς παράγοντες υιοθέτησης του σημερινού αποκεντρωμένου τοπίου είναι το υψηλό κόστος κρατήσεων (\en{gas fees})  που συνδέεται με ορισμένες τεχνολογίες \en{blockchain}, όπως το \en{Ethereum}. Η μελλοντική έρευνα θα μπορούσε να εμβαθύνει σε \en{blockchains} που προσφέρουν χαμηλότερα τέλη συναλλαγών, διασφαλίζοντας ότι το αποκεντρωμένο υπολογιστικό νέφος παραμένει προσιτό σε ένα ευρύτερο κοινό. Και εδώ θα μπορούσε να χρησιμοποιηθεί το \en{Layer 2} του \en{Ethereum} με την χρήση \en{Optimistic Rollups} \cite{ref24} και να δοκιμαστούν περισσότερο οι επιδόσεις σε αλυσίδες όπως το \en{Polygon} που χρησιμοποιήθηκε στην παρούσα εργασία.
\item Επικοινωνία με τον έξω κόσμο: Καθώς τα \en{smart contracts} είναι ντετερμινιστικά, δεν επικοινωνούν ούτε επηρεάζονται από τον έξω κόσμο με \en{real-time} δεδομένα. Για την επίλυση αυτού του περιορισμού θα μπορούσαν να χρησιμοποιηθούν \en{Oracles} όπως το \en{Chainlink} \cite{ref48}, τα οποία μέσω ενός δικού τους \en{blockchain} χρησιμοποιούνται ως αποκεντρωμένα \en{API} μεταξύ των \en{smart contracts} και του έξω κόσμου. Με τον τρόπο αυτό θα μπορούσαν οι διαπραγματεύσεις για την εκτέλεση των εργασιών, αντί για \en{Gwei}, να γίνονται σε δολάρια, ευρώ ή οποιοδήποτε άλλο νόμισμα, και το \en{smart contract} να είναι υπεύθυνο για την μετατροπή των ισοτιμιών.
\item Πλήρης αποκέντρωση: Για πλήρη αποκέντρωση, μπορούν τα \en{apps} που αφορούν το \en{front end} να γίνουν \en{deploy} στο \en{Swarm}, ένα \en{peer-to-peer} σύστημα αποθήκευσης αρχείων, αναπτυγμένο από το \en{Ethereum Foundation}, το οποίο επιτρέπει και την πρόσβαση σε ιστοσελίδες, αντικαθιστώντας τους κεντρικούς \en{servers} \cite{ref45}.
\item Αλγόριθμος βαθμολόγησης επιδόσεων: Περαιτέρω μελέτη μπορεί να πραγματοποιηθεί και για την εύρεση ενός αλγορίθμου υπολογισμού της επίδοσης και φερεγγυότητας πελατών και παρόχων, ο οποίος θα αντιπροσωπεύει πιο ολοκληρωμένα την συμπεριφορά τους.
\item Ευελιξία: Ιδιαίτερο χρήσιμο και αποδοτικό θα ήταν να προστεθούν και άλλες επιλογές για την γλώσσα ανάπτυξης των υπολογιστικών εργασιών εκτός της \en{Java}.
\end{itemize}





\chapter{Εισαγωγή} 
\InitialCharacter{Σ}την σύγχρονη ψηφιακή εποχή, το υπολογιστικό νέφος 
\en{(cloud computing)} έχει αναδειχθεί σε ακρογωνιαίο λίθος της υποδομής
πληροφοριακών συστημάτων, προσφέροντας κλιμακούμενους υπολογιστικούς
πόρους κατά παραγγελία. Παραδοσιακά, οι υπηρεσίες υπολογιστικού νέφους 
κυριαρχούνται από τεχνολογικούς κολοσσούς, όπως οι \en{Amazon 
(Amazon Web Services)}, \en{IBM (IBM Cloud)}, \en{Microsoft (Microsoft Azure)} και 
\en{Google (Google Cloud)}. Αυτά τα συγκεντρωτικά μοντέλα, αν 
και αποτελεσματικά, συχνά στερούνται διαφάνειας στην πολιτική τιμολόγησης, 
την κατανομή των πόρων και τις διαδικασίες λήψης των αποφάσεων. 
Επιπλέον, εισάγουν πιθανές μονοπωλιακές  συμπεριφορές στην τιμολόγηση
και την προσφορά υπηρεσιών.
 
Καθώς ο κόσμος κινείται προς ένα πιο αποκεντρωμένο πρότυπο σε διάφορους 
τομείς, από τα χρηματοοικονομικά έως την αλυσίδα εφοδιασμού, το πεδίο 
υπολογιστικού νέφους δεν αποτελεί εξαίρεση. Η υπόσχεση της αποκέντρωσής 
του προσφέρει την δυνατότητα μεγαλύτερης διαφάνειας, ενισχυμένης ασφάλειας 
και δικαιότερης κατανομής των πόρων. Εξαλείφοντας του μεσάζοντες και 
αξιοποιώντας τους εγγενείς μηχανισμούς εμπιστοσύνης της τεχνολογίας 
\en{Blockchain}, το αποκεντρωμένο υπολογιστικό νέφος μπορεί να εκδημοκρατίσει 
την πρόσβαση στην υπολογιστική ισχύ και να προωθήσει μια ανταγωνιστική 
αγορά.


\section{Αντικείμενο της διπλωματικής}
Η παρούσα εργασία παρουσιάζει μια νέα προσέγγιση για την αποκέντρωση 
των υπηρεσιών υπολογιστικού νέφους μέσω της ανάπτυξης μια αποκεντρωμένης 
εφαρμογής \en{(decentralized App – dApp)} στο \en{Ethereum Blockchain}. 
Το προτεινόμενο σύστημα επιτρέπει στους χρήστες να αναθέτουν υπολογιστικές 
εργασίες γραμμένες σε \en{Java} σε ένα αποκεντρωμένο δίκτυο παρόχων. 
Μέσω ενός μηχανισμού δημοπρασιών, οι πάροχοι των υπολογιστικών πόρων 
υποβάλλουν προσφορές για την ανάληψη εργασιών και οι πελάτες επιλέγουν 
τους παρόχους που επιθυμούν βάσει ενός συνδυασμού της προσφοράς που έχει 
κατατεθεί και του ιστορικού των επιδόσεων τους. Το σύστημα διασφαλίζει την 
ακεραιότητα της εκτέλεσης των εργασιών και προσφέρει μια ασφαλή διαδικασία 
πληρωμής μετά την επιτυχή ολοκλήρωσή τους.

Τα επόμενα κεφάλαια θα εμβαθύνουν στον σχεδιασμό, την αρχιτεκτονική και 
τις λεπτομέρειες υλοποίησης της \en{dApp}, αξιολογώντας τις δυνατότητές της 
να αναδιαμορφώσει το τοπίο του υπολογιστικού νέφους και να προσφέρει ένα 
πιο διαφανές, αξιόπιστο και αποτελεσματικό σύστημα για την ανάθεση 
υπολογιστικών εργασιών σε τρίτους.



\section{Οργάνωση του τόμου}
Η παρούσα εργασία είναι οργανωμένη σε επτά κεφάλαια. Στο κεφάλαιο 2 δίνεται το θεωρητικό υπόβαθρο του \en{cloud computing} και του \en{Blockchain}, με έμφαση στο \en{Ethereum} και τις λύσεις που προσφέρει. Στο κεφάλαιο 3 παρουσιάζεται η προεπισκόπηση, ο σχεδιασμός και αρχιτεκτονική του προτεινόμενου συστήματος. Στα κεφάλαια 4 και 5 αναλύονται οι επιμέρους μηχανισμοί των δημοπρασιών και εκτέλεσης των εργασιών αντίστοιχα, ενώ στο κεφάλαιο 6 περιγράφεται η υλοποίηση του συστήματος. Τέλος, στο κεφάλαιο 7 πραγματοποιείται η πειραματική αξιολόγηση της εφαρμογής και στο κεφάλαιο 8 παρουσιάζονται τα συμπεράσματα, η σημασία τις εργασία, καθώς και οι μελλοντικές επεκτάσεις της.
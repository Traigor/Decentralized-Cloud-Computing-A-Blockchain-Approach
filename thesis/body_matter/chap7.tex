\chapter{Πειραματική αξιολόγηση}
\InitialCharacter{Σ}το κεφάλαιο αυτό αξιολογείται πειραματικά η επίδοση της \en{dApp} ως προς το κόστος και την καθυστέρηση που προσθέτει στην εκτέλεση των εργασιών.


\section{Σύγκριση εκτέλεσης εργασίας με χρήση της \en{dApp} και τοπικά}
Για να συγκριθεί η επίδοση της \en{dApp} και το \en{overhead} που προσθέτει στον υπολογισμό εργασιών, στο κεφάλαιο αυτό εκτελείται ως παράδειγμα η συνάρτηση που παρατίθεται στο \textbf{Παράρτημα B'}.

\subsection{Επιπρόσθετο κόστος εκτέλεσης της εργασίας}
Για την σύγκριση του κόστους που προσθέτει η \en{dApp} σε σχέση με τον τοπικό υπολογισμό μιας εργασίας, εκτελέστηκε η ροή επιτυχημένης εκτέλεσής της στο \en{ehereum} δίκτυο του {hardhat} που προσομοιώνει το \en{Ethereum}, στο \en{Sepolia Testnet (Ethereum)} και στο \en{Mumbai Testnet (Polygon)}.

Τα αποτελέσματα της εκτέλεσης, ως προς το συνολικό κόστος που προστίθεται για καθέναν από τον \en{deployer}, τον \en{client} και τον \en{provider} σε κάθε περίπτωση είναι τα εξής:
\begin{itemize}
    \item \textbf{\en{Hardhat Ethereum}}:
        \begin{itemize}
            \item[-] \en{Deployer}: 0.283491576 \en{ETH} = 478.09 \$
            \item[] \en{Client}:  0.016952904 \en{ETH} = 28.59 \$
            \item[] \en{Provider}: 0.023493096 \en{ETH}  = 39.62 \$
        \end{itemize} 
    \item \textbf{\en{Sepolia Testnet}}:
        \begin{itemize}
            \item[-] \en{Deployer}:  0.01474686 \en{ETH} = 24.71 \$
            \item[] \en{Client}:  0.00084412 \en{ETH} = 1.42 \$
            \item[] \en{Provider}:  0.00106625 \en{ETH} = 1.80 \$
        \end{itemize} 
    \item \textbf{\en{Mumbai Testnet}}:
        \begin{itemize}
            \item[-] \en{Deployer}:  0.11896957549 \en{MATIC} = 0.07 \$
            \item[] \en{Client}:  0.0009054505 \en{MATIC} = 0 \$
            \item[] \en{Provider}:  0.00150822329 \en{MATIC} = 0 \$
        \end{itemize} 
\end{itemize}
όπου η ισοτιμία των \en{ETH} και \en{MATIC} σε δολάρια (\$) είναι με βάση τις τιμές της 21/10/2023.

\begin{table}[h]
    \centering
    \small  % Decrease the font size
    \begin{tabular}{|l|c|c|c|}
    \hline
    & \textbf{\en{Deployer}} & \textbf{\en{Provider}} & \textbf{\en{Client}} \\ \hline
    \textbf{\en{Hardhat Ethereum}} & 478.09 \$ & 28.59 \$ &  39.62 \$ \\ \hline
    \textbf{\en{Sepolia Testnet}} & 24.71 \$ & 1.42 \$& 1.80 \$ \\ \hline
    \textbf{\en{Mumbai Testnet}} & 0.07 \$ &  0.00 \$ &0.00 \$ \\ \hline
    \end{tabular}
    \caption{\en{Gas Fees}}
\end{table}

\subsection{Αύξηση συνολικού χρόνου εκτέλεσης της εργασίας}
Ο κώδικας της εργασίας εκτελέστηκε μέσω της \en{dApp} και τοπικά και τα αποτελέσματα των χρόνων  παρουσιάζονται στο παρακάτω διάγραμμα. Σημειώνεται ότι στον χρόνο που αφορά την \en{dApp} συμπεριλαμβάνονται και οι απαραίτητες αλληλεπιδράσεις του χρήστη και του πελάτη στην πλατφόρμα, οι οποίες ως κατά κύριο λόγο χειροκίνητες μπορεί να εισάγουν καθυστέρηση. Επίσης, για στο συγκεκριμένο παράδειγμα, τόσο εκ μέρους του παρόχου όσο και από την πλευρά του πελάτη, η εργασία εκτελέστηκε στο ίδιο μηχάνημα \en{(Macbook Air M1 2020)}. Κατά κύριο λόγο, οι πάροχοι θα έχουν μηχανήματα με καλύτερες επιδόσεις από αυτά των πελατών, οπότε και οι χρόνοι που αφορούν την εκτέλεση της εργασίας θα είναι μειωμένοι. 


\begin{center}

    \begin{tikzpicture} \centering
    \begin{axis}[
        xlabel=Εκτέλεση Εργασίας,
        xtick=\empty,
        ylabel=Χρόνος Εκτέλεσης \en{(sec)},
        enlargelimits=0.25,
        enlarge x limits=false,
        legend style={at={(.98,.98)},anchor=north east},
        ybar interval=0.7,
        ymin = 200
    ]
    \addplot coordinates {(1,660) (2,660)};
    \addplot coordinates {(1,517) (2,517)};

    \legend{\en{dApp},\en{Locally}}
    \end{axis}
    \end{tikzpicture}
    \end{center}

\section{Σχολιασμός Αποτελεσμάτων}
Μετά την ολοκλήρωση της εκτέλεσης των πειραμάτων παρατηρούμε ότι το κόστος που η εφαρμογή προσθέτει λόγω χρεώσεων του \en{Blockchain} εξαρτάται από την πλατφόρμα που θα επιλεχθεί, ενώ ο χρόνος που προστίθεται δεν είναι πολύς.

Ως προς το κόστος, είναι εμφανές ότι οι χρεώσεις είναι πολύ αυξημένες στην περίπτωση του \en{Ethereum}, ακόμα και μέσω προσομοίωσης. Η τιμή των επιπρόσθετων χρεώσεων που απαιτεί το \en{Blockchain} εξαρτάται από την συμφόρηση στο δίκτυο και το πλήθος συναλλαγών στην αλυσίδα. Για τον λόγο αυτό, παρατηρούνται και στο \en{Sepolia Testnet} τόσο σημαντικά μικρότερες χρεώσεις σε σχέση με το \en{Ethereum}, καθώς επειδή χρησιμοποιείται για την μελέτη εφαρμογών, έχει σαφώς λιγότερους χρήστες. Μια πρόταση για την μείωση του σημαντικού κόστους των χρεώσεων του \en{Ethereum Blockchain} είναι η χρήση πλατφορμών του \en{Layer 2}, όπως το \en{Polygon}, η λειτουργία του οποίου περιγράφηκε σε προγούμενο κεφάλαιο. Παρότι δεν υπάρχει τρόπος να λάβουμε μετρήσεις που να προσομοιώνουν τις τιμές που θα προκύψουν εκεί, μπορούμε να λάβουμε ένα δείγμα από την διαφορά που παρατηρείται στις χρεώσεις στο \en{Mumbai Testnet}, το οποίο αφορά την πειραματική ανάπτυξη εφαρμογών στο \en{Polygon}, σε σχέση με το \en{Sepolia Testnet} που αφορά το \en{Ethereum}.

Ως προς το κόστος, ο επιπλέον χρόνος που παρατηρείται στην εκτέλεση της εφαρμογής δεν κρίνεται αποτρεπτικός, Αυτό βασίζεται στο γεγονός ότι η εκτέλεση για την σύγκριση των χρόνων έγινε στο πειραματικό \en{Sepolia Testnet} που δεν έχει την ίδια επίδοση με το \en{Ethereum Mainnet}. Ακόμη, έναν λόγο που αυξάνεται η συνολική χρονική διάρκεια εκτέλεσης της εργασίας αποτελεί το γεγονός ότι σε αυτήν περιλαμβάνονται και οι αλληλεπιδράσεις των χρηστών με την εφαρμογή, όπως για παράδειγμα η χειροκίνητη επιλογή τιμής προσφοράς από τον πάροχο και η επιλογή παρόχου από τον πελάτη. Για επιτάχυνση της διαδικασίας μπορεί να υπάρξουν επιλογές αυτοματοποίησης των αλληλεπιδράσεων αυτών, όπως ο πάροχος να προσφέρει πάντα κάποια συγκεκριμένη τιμή και ο πελάτης να επιλέγει αυτόματα την καλύτερη προσφορά. Τέλος, στο παρόν πείραμα η εργασία εκτελέστηκε και στις δύο περιπτώσεις στο ίδιο μηχάνημα, ενώ κατά κύριο λόγο οι πάροχοι θα προσφέρουν καλύτερη υποδομή από αυτή που έχουν οι πελάτες. 
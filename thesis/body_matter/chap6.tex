\chapter{Υλοποίηση του συστήματος}
\InitialCharacter{H} ανάπτυξη της αποκεντρωμένης εφαρμογής (\en{dApp}) περιλαμβάνει ένα συνδυασμό τεχνολογιών αιχμής και καινοτόμων λύσεων για την αντιμετώπιση των προκλήσεων του αποκεντρωμένου υπολογιστικού νέφους. Αυτό το κεφάλαιο εμβαθύνει στις τεχνικές ιδιαιτερότητες της \en{dApp}, τα εργαλεία και τις πλατφόρμες που χρησιμοποιήθηκαν και τις προκλήσεις που αντιμετωπίστηκαν κατά την ανάπτυξή της.

\section{Τεχνικές ιδιαιτερότητες της \en{dApp}}
Η αρχιτεκτονική της \en{dApp} συνδυάζει την τεχνολογία του \en{Blockchain}, του \en{containerization} και των κατανεμημένων συστημάτων αρχείων, εξασφαλίζοντας ένα αποτελεσματικό και ασφαλές σύστημα.

\begin{enumerate}
    \item Έξυπνα συμβόλαια: Αναπτυγμένα με τη χρήση της \en{Solidity}, τα έξυπνα συμβόλαια \en{AuctionsManager} και \en{TasksManager} αναπτύσσονται στα \en{blockchains} \en{Ethereum} και \en{Polygon}. Διαχειρίζονται τη διαδικασία δημοπρασίας, τον κύκλο ζωής των εργασιών, τις πληρωμές και τα χρηματικές εγγυήσεις.
    \item Ενσωμάτωση του \en{IPFS}: Το Διαπλανητικό Σύστημα Αρχείων (\en{IPFS}) είναι ένα αποκεντρωμένο σύστημα αποθήκευσης, με βάση την διεύθυνση του περιεχομένου, το οποίο διανέμει αποθηκευμένα αρχεία μεταξύ ομότιμων χρηστών σε ένα δίκτυο \en{P2P}. Το περιεχόμενο κάθε αρχείου κατακερματίζεται και ο κατακερματισμός αυτός \en{(Content Identifier - CID)} χρησιμοποιείται για την ταυτοποίηση και ανάκτηση του συγκεκριμένου αρχείου \cite{ref42}.
    Το σύστημα της εργασίας χρησιμοποιεί το \en{IPFS} για την αποθήκευση και ανάκτηση των μεταγλωττισμένων κλάσεων \en{Java} και των αποτελεσμάτων της εκτέλεσης. Αυτό διασφαλίζει την αμεταβλητότητα των δεδομένων και την αποκεντρωμένη πρόσβαση.
    \item \en{Docker Containerization}: Το \en{Docker} χρησιμοποιείται για τη δημιουργία απομονωμένων περιβαλλόντων για την εκτέλεση εργασιών, εξασφαλίζοντας σταθερή απόδοση και ασφάλεια έναντι πιθανών απειλών.
    \item Γραφική διεπαφή της εφαρμογής: Αναπτυγμένη με τη χρήση του \en{React}, η γραφική διεπαφή διευκολύνει τις αλληλεπιδράσεις του χρήστη, από την έναρξη δημοπρασίας έως την παρακολούθηση εργασιών.
\end{enumerate}

\section{Εργαλεία, γλώσσες προγραμματισμού και πλατφόρμες ανάπτυξης}
Για την ανάπτυξη της \en{dApp} αξιοποιήθηκε πλήθος εργαλείων και πλατφορμών με σκοπό την βέλτιστη απόδοση και εμπειρία των χρηστών.

\begin{enumerate}
    \item Γλώσσες προγραμματισμού
    \begin{itemize}
        \item[-] \en{Solidity}: Για την ανάπτυξη έξυπνων συμβολαίων στο \en{blockchain} του \en{Ethereum} και του \en{Polygon}. 
        \item[-] \en{Java}: Για την υλοποίηση της υπολογιστικής εργασίας τους πελάτη.
        \item[-] \en{Javascript}: Για την γραφική διεπαφή (\en{frontend}) της \en{dApp}.
        \item[-] \en{Typescript}: Για τις λειτουργίες του \en{backend} της \en{dApp}.
    \end{itemize}
    \item Εργαλεία και πλατφόρμες ανάπτυξης
    \begin{itemize}
        \item[-] \en{Hardhat}: Περιβάλλον ανάπτυξης που χρησιμοποιείται για την ανάπτυξη εφαρμογών στο \en{Ethereum blockchain} \cite{ref44}.
        \item[-] \en{Metamask}: Επέκταση (\en{plugin}) του προγράμματος περιήγησης (\en{browser}) που λειτουργεί ως πορτοφόλι για το \en{Ethereum} και επιτρέπει οικονομικές αλληλεπιδράσεις με την \en{dApp} \cite{ref43}.
        \item[-] \en{Ganache}: Προσωπικό τοπικό \en{blockchain} που χρησιμοποιείται για το \en{testing} κατά την ανάπτυξη εφαρμογών στο \en{Ethereum}.
        \item[-] \en{Web3.js}: Βιβλιοθήκη της \en{Javascript} που επιτρέπει τις αλληλεπιδράσεις μεταξύ κόμβων του \en{Ethereum blockchain}.
        \item[-] \en{React}: \en{Framework} της \en{Javascript} που χρησιμοποιείται για την ανάπτυξη της γραφικής διεπαφής εφαρμογών.
        \item[-] \en{Testnets}: Για σκοπούς πειραματισμού και ελέγχου ορθής λειτουργίας, η \en{dApp} αναπτύχθηκε στα πειραματικά \en{live blockchains} (\en{testnets}) του \en{Ethereum} (\en{Sepolia testnet}) και του \en{Polygon} (\en{Mumbai testnet}), τα οποία προσομοιώνουν την λειτουργία των πραγματικών αντίστοιχων \en{blockchains} (\en{Mainnets}). Για την πρόσβαση στα \en{testnets}, χρησιμοποιήθηκε και την ανάπτυξη της εφαρμογής σε αυτά χρησιμοποιήθηκε η πλατφόρμα ανάπτυξης \en{Alchemy} \cite{ref52,ref53,ref54}. 
    \end{itemize}
\end{enumerate}

\section{Προκλήσεις και λύσεις}
Κατά την ανάπτυξη της \en{dApp}, αντιμετωπίστηκαν οι εξής προκλήσεις:
\begin{enumerate}
    \item Αμεταβλητότητα των δεδομένων: Η διασφάλιση ότι οι υπολογιστικές εργασίες παρέμεναν αμετάβλητες ήταν πρωταρχικής σημασίας. Η πρόκληση αντιμετωπίστηκε με την ενσωμάτωση του \en{IPFS}, το οποίο εξασφαλίζει τον αμετάβλητο χαρακτήρα των δεδομένων και παρέχει αποκεντρωμένη αποθήκευση.
    \item Μηχανισμός επαλήθευσης: Ο σχεδιασμός ενός ελαφρύ αλλά αποτελεσματικού μηχανισμού επαλήθευσης αποτέλεσε πρόκληση. Η λύση ήταν η εισαγωγή της μεθόδου \en{getVerification}, η οποία παρέχει μια απλή αλλά ισχυρή διαδικασία επαλήθευσης.
    \item Συνέπεια του περιβάλλοντος εκτέλεσης: Πρόκληση αποτέλεσε και η εξασφάλιση ενός συνεπούς περιβάλλοντος εκτέλεσης σε διαφορετικούς παρόχους. Η λύση ήταν το \en{Docker containerization}, το οποία προσέφερε ένα απομονωμένο και συνεπές περιβάλλον για την εκτέλεση κάθε εργασίας.
    \item Ανησυχίες σχετικά με την ασφάλεια: Η προστασία του συστήματος του παρόχου από πιθανό κακόβουλο κώδικα ήταν μια σημαντική πρόκληση. Η προσέγγιση του \en{Docker containerization}, σε συνδυασμό με το απομονωμένο περιβάλλον εκτέλεσης, αντιμετώπισε αποτελεσματικά και αυτή την ανησυχία.
    \item Κόστος συναλλαγών: Οι συναλλαγές στο \en{Ethereum} συνοδεύονται από κόστος σε \en{gas}. Η βελτιστοποίηση των λειτουργιών των έξυπνων συμβολαίων ήταν απαραίτητη για την ελαχιστοποίηση αυτών των εξόδων, εξασφαλίζοντας την όσο το δυνατόν πιο προσιτή τιμή για τους χρήστες.
\end{enumerate}

Εν κατακλείδι, για την υλοποίηση της \en{dApp} παρουσιάστηκαν πολλές προκλήσεις, η επίλυση των οποίων ήταν απαραίτητη για την απόδειξη των δυνατοτήτων του αποκεντρωμένου υπολογιστικού νέφους. Ο συνδυασμός τεχνολογιών και οι στρατηγικές λύσεις που χρησιμοποιήθηκαν κατέληξαν σε μια πλατφόρμα που υπόσχεται πραγματική εκτέλεση, ασφάλεια και διαφάνεια στο πεδίο της αποκεντρωμένης πληροφορικής.
\chapter{Μηχανισμός δημοπρασίας}
\InitialCharacter{Ο} μηχανισμός δημοπρασίας χρησιμεύει ως ο ακρογωνιαίος λίθος της προτεινόμενης αποκεντρωμένης εφαρμογής (\en{dApp}), δημιουργώντας μια διαφανή και δίκαιη πλατφόρμα για την κατανομή των υπολογιστικών εργασιών. Η παρούσα ενότητα διευκρινίζει την περίπλοκη δυναμική της διαδικασίας δημοπρασίας, τις στρατηγικές υποβολής προσφορών που χρησιμοποιούν οι πάροχοι και τα πολύπλευρα κριτήρια που καθοδηγούν την επιλογή ενός παρόχου από έναν πελάτη.

\section{Επισκόπηση της διαδικασίας δημοπρασίας}
Η διαδικασία δημοπρασίας ξεκινά όταν ένας πελάτης υποβάλλει μια υπολογιστική εργασία στην \en{dApp}. Η διαδικασία αυτή είναι δομημένη ώστε να διασφαλίζεται η δικαιοσύνη, η διαφάνεια και η βέλτιστη κατανομή εργασιών:
\begin{enumerate}
    \item Υποβολή εργασίας: Οι πελάτες ξεκινούν τη δημοπρασία αναφέροντας τις λεπτομέρειες της υπολογιστικής εργασίας και συγκεκριμένα την αναμενόμενη προθεσμία ολοκλήρωσης και τη σχετική συμβολοσειρά επαλήθευσης.
    \item Υποβολή προσφορών: Μετά την υποβολή της εργασίας, οι πάροχοι έχουν την ευκαιρία να εξετάσουν τις λεπτομέρειες της εργασίας και να υποβάλλουν τις προσφορές τους με την προτεινόμενη τιμή τους (σε \en{wei} ανά δευτερόλεπτο εκτέλεσης της εργασίας).
    \item Επιλογή παρόχου: Όσο η δημοπρασία παραμένει ενεργή, ο πελάτης μπορεί να επιλέξει τον πάροχο που επιθυμεί, βασιζόμενος σε διάφορους παράγοντες που θα αναλυθούν παρακάτω. Η απόφαση αυτή σηματοδοτεί την ολοκλήρωση της δημοπρασίας και την έναρξη της φάσης εκτέλεσης της εργασίας.
\end{enumerate}

\section{Μηχανισμός βαθμολόγησης}
Στο \en{smart contract} \en{TasksManager} αποθηκεύεται το ιστορικό των επιδόσεων των παρόχων και των πελατών στην διαδικασία μέσω ενός μηχανισμού βαθμολόγησης. Ο μηχανισμός αυτός είναι καθοριστικής σημασίας για τη διατήρηση της ακεραιότητας της \en{dApp}, διασφαλίζοντας ότι τόσο οι πάροχοι όσο και οι πελάτες λογοδοτούν για τις πράξεις και τις επιδόσεις τους.
\begin{itemize}
    \item Βαθμολογία του παρόχου: Σε κάθε πάροχο αποδίδεται μια βαθμολογία, η οποία προκύπτει από τον συνδυασμό των ανεπιτυχών (\en{downVotes}) και επιτυχών (\en{upVotes}) εκτελέσεων εργασιών, η οποία αντικατοπτρίζει το ιστορικό των επιδόσεων τους. Ο υπολογισμός της βαθμολογίας χρησιμοποιεί τη μεθοδολογία "ταξινόμησης εμπιστοσύνης" που βασίζεται στο διάστημα βαθμολογίας \en{Wilson}, έναν διάσημο αλγόριθμο που χρησιμοποιείται από πλατφόρμες όπως το \en{Reddit} για την ταξινόμηση των σχολίων στην πλατφόρμα. Με τον τρόπο αυτό διασφαλίζεται ότι η βαθμολογία δεν είναι απλώς ένας μέσος όρος, αλλά μια αναπαράσταση της αξιοπιστίας του παρόχου με βάση τον όγκο και την αναλογία των \en{upvotes} προς τα \en{downvotes} του. 
    \item Βαθμολογία πελάτη: Στους πελάτες αποδίδεται επίσης μια αντίστοιχη βαθμολογία, ενδεικτική των ιστορικών αλληλεπιδράσεών τους και της συνέπειάς τους στην αμοιβή των παρόχων για επιτυχώς εκτελεσμένες εργασίες. Η βαθμολογία αυτή παίζει καθοριστικό ρόλο στη διαμόρφωση της στρατηγικής προσφορών του παρόχου.
\end{itemize}

\section{Κριτήρια που καθοδηγούν τον πελάτη στην επιλογή \\παρόχου}
Η επιλογή του παρόχου από τον πελάτη επηρεάζεται από μια σειρά παραγόντων, διασφαλίζοντας ότι η απόφαση είναι τόσο τεκμηριωμένη όσο και βέλτιστη:
\begin{itemize}
    \item Προσφορά τιμής: Η προτεινόμενη τιμή του παρόχου παραμένει πρωταρχικής σημασίας, με τους πελάτες να κλίνουν φυσικά προς τις ανταγωνιστικές τιμές.
    \item Το ιστορικό επιδόσεων του παρόχου: Η βαθμολογία του παρόχου, όπως υπολογίζεται με την προαναφερθείσα μεθοδολογία, προσφέρει πληροφορίες σχετικά με την αξιοπιστία και τις προηγούμενες επιδόσεις του.
    \item Προηγούμενες συνεργασίες: Προηγούμενες συνεργασίες με τον συγκεκριμένο πάροχο που κατέληξαν σε θετικά αποτελέσματα μπορούν να επηρεάσουν σημαντικά την απόφαση ενός πελάτη.
\end{itemize}

\section{Παράγοντες που διαμορφώνουν τη στρατηγική προσφορών του παρόχου}
Οι πάροχοι, κατά τη διαμόρφωση των προσφορών τους, εξετάζουν πληθώρα παραγόντων για να βελτιστοποιήσουν τις πιθανότητές τους να επιλεγούν, εξασφαλίζοντας παράλληλα την κερδοφορία τους:
\begin{itemize}
    \item Η αξιοπιστία του πελάτη: Η βαθμολογία ενός πελάτη, ενδεικτική των προηγούμενων αλληλεπιδράσεών του και της συνέπειας των πληρωμών του, μπορεί να επηρεάσει το ποσό της προσφοράς του παρόχου.
    \item Διαθεσιμότητα πόρων του παρόχου: Οι πάροχοι με άφθονους πόρους ενδέχεται να έχουν την προδιάθεση να υποβάλουν χαμηλότερη προσφορά για να εξασφαλίσουν την εργασία.
    \item Ιστορικό αλληλεπιδράσεων με τον πελάτη: Οι θετικές αλληλεπιδράσεις του παρελθόντος με τον πελάτη μπορούν να παρακινήσουν έναν πάροχο να υποβάλει μια πιο ανταγωνιστική προσφορά.
\end{itemize}

Συνοψίζοντας, ο μηχανισμός δημοπρασίας, ο οποίος υποστηρίζεται από τη διαφανή διαδικασία υποβολής προσφορών, τα κριτήρια αξιολόγησης και το σύστημα βαθμολόγησης, ενισχύει τη δέσμευση της \en{dApp} για την προώθηση ενός αξιόπιστου και αποτελεσματικού περιβάλλοντος για την αποκεντρωμένη εφαρμογή υπολογιστικής νέφους.

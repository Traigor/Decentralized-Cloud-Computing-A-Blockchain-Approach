\chapter{Θεωρητικό υπόβαθρο}
\InitialCharacter{Σ}το κεφάλαιο αυτό παρουσιάζονται η εξέλιξη του 
\en{Cloud Computing}, η τεχνολογία \en{Blockchain}, καθώς και οι
αποκεντρωμένες εφαρμογές. Επίσης, γίνεται αναφορά σε προηγούμενες εργασίες 
που ασχολούνται με την ιδέα του αποκεντρωμένου υπολογιστικού νέφους.

\section{\en{Cloud Computing}}
\subsection{Εξέλιξη \en{Cloud Computing}}
Το υπολογιστικό νέφος αναφέρεται στην παροχή διαφόρων υπηρεσιών μέσω του 
διαδικτύου, συμπεριλαμβανομένων της αποθήκευσης και της υπολογιστικής ισχύος. 
Η έννοια, αν και έγινε ευρέως διαδεδομένη τα τελευταία χρόνια, έχει τις ρίζες 
της στην δεκαετία του 1960 με την έλευση του \en{utility computing}. Η ιδέα 
βασιζόταν στο εγχείρημα ότι οι υπολογιστικοί πόροι, όπως το νερό και το 
ηλεκτρικό ρεύμα, μπορούν να παρέχονται σε οικίες και επιχειρήσεις ως υπηρεσία 
κοινής ωφέλειας. Ωστόσο, μόλις την δεκαετία του 2000, με την άνοδο του 
διαδικτύου υψηλής ταχύτητας, τις σημαντικές εξελίξεις στις τεχνολογίες
της εικονικοποί-ησης \en{(virtualization)} και την αύξηση της υπολογιστικής ισχύος, 
το \en{cloud computing} άρχισε να παίρνει την σύγχρονη μορφή του.

Εταιρείες όπως η \en{Amazon}, η \en{Microsoft} και η \en{Google}, ήταν από τις 
πρώτες που αναγνώρισαν τις δυνατότητες ενοικίασης των τεράστιων υπολογιστικών 
πόρων τους. Η \en{Amazon Web Services (AWS)} ξεκίνησε το 2006, σηματοδοτώντας 
την έναρξη της σύγχρονης εποχής του νέφους. Τα επόμενα χρόνια παρατηρήθηκε 
μια έκρηξη υπηρεσιών \en{cloud computing}, οι οποίες καλύπτουν διάφορες τεχνολογικές ανάγκες, 
από την υποδομή ως υπηρεσία \en{(Infrastructure as a Service – IaaS)} έως το 
λογισμικό ως υπηρεσία \en{(Software as a Service – SaaS)}.

\subsection{Υφιστάμενα συγκεντρωτικά μοντέλα και οι περιορισμού τους}
Τα συγκεντρωτικά μοντέλα υπολογιστικού νέφους, προσφερόμενα από τεχνολογικούς 
κολοσσούς όπως το \en{AWS}, το \en{Google Cloud} και το \en{Microsoft Azure}
κυριαρχούν στην τρέχουσα αγορά. Οι υπηρεσίες αυτές παρέχουν στους χρήστες 
αξιόπιστες, κλιμακούμενες και συχνά οικονομικά αποδοτικές λύσεις. 

Ωστόσο, συνοδεύονται από εγγενείς περιορισμούς:
\begin{itemize}
\item Έλλειψη διαφάνειας: Ο τρόπος τιμολόγησης μπορεί να πολύπλοκος και οι 
χρήστες συχνά δεν έχουν σαφή εικόνα για την κατανομή των πόρων.
\item Ενιαίο σημείο αποτυχίας (\en{single point of failure}): Τα συγκεντρωτικά 
συστήματα, εκ κατασκευής, έχουν πιθανά σημεία συμφόρησης. Εάν ένας μεγάλος 
πάροχος υπηρεσιών υπολογιστικού νέφους αντιμετωπίσει διακοπή λειτουργίας, 
αυτό μπορεί να επηρεάσει εκατομμύρια χρήστες.
\item Ανησυχίες σχετικά με το απόρρητο των δεδομένων: Με τα δεδομένα 
συγκεντρωμένα σε λίγες εταιρείες, υπάρχουν βάσιμες ανησυχίες σχετικά με την 
κατάχρηση, την παρακολούθησή τους, καθώς και τις παραβιάσεις των υπάρχοντων 
συστημάτων.
\item Δυνατότητα μονοπωλιακής συμπεριφοράς: Η κυριαρχία λίγων εταιρειών στην 
αγορά μπορεί να οδηγήσει σε μείωση του ανταγωνισμού και της καινοτομίας.
\end{itemize}
%//TODO: about environment too 

%//TODO: Add more about bitcoin, blockchain, ethereum, smart contracts, dApps, layers above ethereum, etc.
\section{\en{Blockchain}}
\subsection{Εισαγωγή στην τεχνολογία \en{Blockchain}}
Το 2008, μια οντότητα - άνθρωπος ή ομάδα ανθρώπων - με το ψευδώνυμο \en{Satoshi Nakamoto} παρουσίασε το \en{Bitcoin}, ένα αποκεντρωμένο ψηφιακό νόμισμα. Πέρα από τη νομισματική του λειτουργία, το \en{Bitcoin} εισήγαγε την τεχνολογία του \en{Blockchain}. 
Το \en{Blockchain} είναι μια αλυσίδα από \en{blocks}, στα οποία αποθηκεύονται οι συναλλαγές που πραγματοποιούνται στο δίκτυο. Αποτελεί, δηλαδή, μια βάση δεδομένων συναλλαγών σε ένα δίκτυο, η οποία λειτουργεί ως αποκεντρωμένο λογιστικό βιβλίο. Προσφέρει διαφάνεια, ασφάλεια και απουσία κεντρικού ελέγχου. Στον πυρήνα του, το \en{blockchain}, χρησιμοποιώντας κρυπτογραφικές αποδείξεις και έναν αλγόριθμο συναίνεσης (\en{consensus algorithm}) είναι ανθεκτικό στην τροποποίηση των δεδομένων του, διατηρώντας με τον τρόπο αυτό την ακεραιότητά τους, και εξασφαλίζει την εμπιστοσύνη μεταξύ των συμμετεχόντων.

\subsection{\en{Ethereum}}
To \en{Ethereum}, το οποίο παρουσιάστηκε το 2013 από τον \en{Vitalik Buterin} και δημοσιεύθηκε το 2015, επέκτεινε την ιδέα του \en{blockchain} και δημιούργησε μια δημόσια πλατφόρμα κατανεμημένου υπολογισμού ανοικτού κώδικα, η οποία διαθέτει την λειτουργικότητα των έξυπνων συμβολαίων. Με τον τρόπο αυτό, παρέχει στους προγραμματιστές την \en{Ethereum Virtual Machine (EVM)}, μια αποκεντρωμένη εικονική μηχανή που είναι \en{Turing complete}, μετατρέποντας τη δημιουργία εφαρμογών στο \en{blockchain} σε πολύ απλούστερη και εύκολη διαδικασία.

\subsection{Εξυπνά συμβόλαια και Aποκεντρωμένες εφαρμογές}
Μία από τις βασικές καινοτομίες του \en{Ethereum} είναι το έξυπνο συμβόλαιο (\en{smart contract}). Τα έξυπνα συμβόλαια είναι ντετερμινιστικά αυτοεκτελούμενα συμβόλαια όπου οι όροι γράφονται απευθείας σε κώδικα και διανέμονται στο \en{blockchain}. Εκτελούν αυτόματα αξιόπιστες συναλλαγές, χωρίς την ανάγκη διαμεσολάβησης τρίτου, με τρόπο ανιχνεύσιμο και μη αναστρέψιμο. Η προσθήκη των έξυπνων συμβολαίων επεκτείνει την χρήσης της τεχνολογίας \en{blockchain} πέρα από τις χρηματοοικονομικές συναλλαγές - όπως αυτές στις οποίες απευθύνεται το \en{Bitcoin} - σε οποιονδήποτε τομέα η εμπιστοσύνη είναι απαραίτητη, όπως τα συστήματα ψηφοφορίας και οι εφαρμογές \en{Internet of Things (IoT)}.

Οι αποκεντρωμένες εφαρμογές, ή \en{dApps}, είναι ένα άμεσο προϊόν της λειτουργικότητας των έξυπνων συμβολαίων, που εκτελούνται σε ένα δίκτυο \en{blockchain} με ομότιμο τρόπο. Αυτές οι εφαρμογές δεν απαιτούν κεντρική αρχή, είναι ανοικτού κώδικα και δίνουν κίνητρα στους χρήστες μέσω κρυπτογραφικών \en{tokens}. Αξιοποιούν τα οφέλη της τεχνολογίας \en{blockchain} για να διασφαλίσουν ότι καμία μεμονωμένη οντότητα δεν έχει τον έλεγχο της εφαρμογής, προσφέροντας ένα νέο επίπεδο ασφάλειας και εμπιστοσύνης για τους χρήστες.

\subsection{\en{Ether} και \en{Gas Fees}}
Το κρυπτονόμισμα του \en{Ethereum}, το \en{Ether} (\en{ETH}), εξυπηρετεί δύο βασικούς σκοπούς: την αποζημίωση των κόμβων του δικτύου για τους υπολογισμούς που εκτελούν και τη διαπραγμάτευσή του ως ψηφιακό νόμισμα σε διάφορα ανταλλακτήρια κρυπτονομισμάτων. Στο δίκτυο του \en{Ethereum}, οι αμοιβές των συναλλαγών μετρώνται με βάση την υπολογιστική πολυπλοκότητα, τη χρήση εύρους ζώνης και τις ανάγκες αποθήκευσης, οι οποίες υπολογίζονται σε όρους \en{gas} και πληρώνονται σε \en{ETH}. Αυτό διασφαλίζει ότι κακόβουλα προγράμματα ή αναποτελεσματικός κώδικας δεν φράσσουν το δίκτυο.

Για να διευκολύνει τις συναλλαγές και τους υπολογισμούς, το \en{Ethereum} χρησιμοποιεί και μονάδες μικρότερης αξίας. Η μικρότερη μονάδα του \en{Ether} είναι γνωστή ως \en{wei}. Ένα \en{Ether} ισοδυναμεί με ένα πεντάκις εκατομμύριο \en{wei} ($1 \en{wei} = 10^{-18} \en{eth}$). H αμέσως επόμενη μονάδα που χρησιμοποιείται ονομάζεται \en{gwei} (\en{Gigawei}) και ισοδυναμεί με ένα δισεκατομμύριο \en{wei} ή 0,000000001 \en{Ether}.

Αυτές οι μικρότερες μονάδες \en{Ether} είναι ζωτικής σημασίας για την ακρίβεια στις συναλλαγές, ειδικά όταν πρόκειται για χρεώσεις \en{gas}, καθώς επιτρέπουν στους χρήστες να καθορίζουν το ακριβές ποσό που είναι διατεθειμένοι να πληρώσουν ανά μονάδα \en{gas}, χωρίς να χρειάζεται να χρησιμοποιούν εξαιρετικά μικρής αξίας δεκαδικά ψηφία. Αυτό το σύστημα όχι μόνο παρέχει μια πιο κατανοητή κλίμακα για τους χρήστες, αλλά επιτρέπει επίσης στο δίκτυο \en{Ethereum} να χειρίζεται τις συναλλαγές και τις αλληλεπιδράσεις έξυπνων συμβολαίων με ακρίβεια, εξασφαλίζοντας δικαιοσύνη για όλα τα εμπλεκόμενα μέρη.

\subsection{\en{Ethereum} 2.0}
Αρχικά, το \en{Ethereum blockchain} χρησιμοποιούσε ως αλγόριθμο συναίνεσης (\en{consensus algorithm}) το \en{Proof of Work (PoW)} που χρησιμοποιεί και το \en{Bitcoin}. Το γεγονός αυτό εισήγαγε περιορισμούς στην επεκτασιμότητα του δικτύου, την ενεργειακή αποδοτικότητά του, καθώς και υψηλές χρεώσεις στις συναλλαγές. Στις 15 Σεπτεμβρίου 2022, ολοκληρώθηκε η μετάβαση στο \en{blockchain Ethereum} 2.0, το οποίο χρησιμοποιεί για \en{consensus algorithm} το \en{Proof of Stake (PoS)}, μειώνοντας έτσι την ενεργειακή κατανάλωση κατά 99\%. 

\subsection{\en{Layer 2}}
Οι τρεις επιθυμητές ιδιότητες της τεχνολογίας \en{blockchain} είναι η αποκέντρωση, η ασφάλεια και η επεκτασιμότητα. Ωστόσο, σύμφωνα με το τρίλημμα του \en{blockchain}, μια απλή αρχιτεκτονική του μπορεί να πετύχει μόνο δύο από τις τρεις. Έτσι, προς επίτευξη μιας ασφαλούς και αποκεντρωμένης αλυσίδας, πρέπει να θυσιαστεί η επεκτασιμότητα. 
Καθώς το \en{Ethereum} επεξεργάζεται σήμερα περισσότερες από 1 εκατομμύριο συναλλαγές την ημέρα, με δυνατότητα επεξεργασίας 15 συναλλαγές το δευτερόλεπτο, η συνεχώς αυξανόμενη ζήτηση για χρήση του μπορεί να προκαλέσει υψηλές χρεώσεις στις συναλλαγές.
Στο σημείο αυτό εισάγονται τα δίκτυα επιπέδου 2 (\en{layer} 2), τα οποία, ενώ χρησιμοποιούν το δίκτυο του επιπέδου 1 (\en{mainnet Ethereum}), παρέχουν τρόπους για επεξεργασία των συναλλαγών εκτός αυτού (\en{off-chain computations}). Έτσι, αφαιρούν το βάρος επιβεβαίωσης κάθε συναλλαγής από το επίπεδο 1, επιτρέποντάς του να γίνει λιγότερο συμφορημένο και όλα να γίνονται πιο κλιμακούμενα, χωρίς να θυσιάζεται η αποκέντρωση και η ασφάλεια. 



\section{Προηγούμενες εργασίες σχετικά με το αποκεντρωμένο \en{cloud computing} και τις \en{dApps}}
Οι περιορισμοί των συγκεντρωτικών μοντέλων υπολογιστικού νέφους και οι 
δυνατότητες της τεχνολογίας \en{blockchain}, οδήγησαν ερευνητές και 
προγραμματιστές να διερευνήσουν το αποκεντρωμένο υπολογιστικό νέφος. 
Έργα όπως το \en{Golem}, το \en{iExec}, το \en{Filecoin} και το \en{SONM} 
έχουν αποτολμήσει το εγχείρημα αυτό, με στόχο την δημιουργία μιας 
αποκεντρωμένης αγοράς υπολογιστικής ισχύος και κατανεμημένου αποθηκευτικού 
χώρου.
Οι πλατφόρμες αυτές επιτρέπουν στους χρήστες να νοικιάζουν τους αδρανείς 
υπολογιστικούς ή αποθηκευτικούς πόρους τους, δημιουργώντας ένα δίκτυο 
ομότιμων κόμβων.

%//TODO: Add
[Προσθήκη σχολίων και διαφορών...]

Το \en{CloudAgora} παρουσιάζεται ως μια πλατφόρμα που στοχεύει να σπάσει 
το μονοπώλιο των παραδοσιακών παρόχων υπολογιστικού νέφους. Αξιοποιεί 
την τεχνολογία \en{blockchain} για να προσφέρει μια αποκεντρωμένη αγορά 
υπολογιστικών πόρων και αποθήκευσης, επιτρέποντας σε οποιονδήποτε δυνητικό 
πάροχο πόρων να αξιοποιεί τους αδρανείς πόρους του και να ανταγωνίζεται με 
τους υπόλοιπους με ίσους όρους. Στην περίπτωση της εκτέλεσης εργασίας με 
υπολογιστικούς πόρους, για την επαλήθευση της ορθής εκτέλεσής της, 
χρησιμοποιεί το \en{TrueBit Protocol} το οποίο αποτελεί μια σύνθετη επιλογή που προσφέρει περιορισμούς, προσθέτοντας καθυστέρηση στην εκτέλεση των εργασιών και κόστος στις συναλλαγές.

To \en{ChainFaas} στοχεύει στην αξιοποίηση της υπολογιστικής ικανότητας 
των προσωπικών υπολογιστών για κατανεμημένους υπολογισμούς, αναθέτοντας 
σε αυτούς εργασίες υπολογιστικής ισχύος. Για την επίτευξη της εμπιστοσύνης, 
χρησιμοποιείται ιδιωτικό (\en{private}) \en{blockchain}, όπου μπορούν να 
συμμετάσχουν σε αυτό μόνο χρήστες που έχουν πάρει άδεια από τον διαχειριστή του δικτύου και μπορούν να ελεγχθούν ανά πάσα στιγμή, σε αντίθεση με τα δημόσια (\en{public}) \en{blockchains} όπου μπορεί να συμμετάσχει ο καθένας χωρίς την δυνατότητα 
ελέγχου.

%//TODO: Add papers, describe better the platforms, especially cloudAgora and chainFaas
[Προσθήκη/ Βελτίωση...]
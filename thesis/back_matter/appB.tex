\chapter{Υπολογιστική Εργασία \en{Java}}
Η \en{Java} εργασία που χρησιμοποιήθηκε στο κεφάλαιο 7 για την πειραματική αξιολόγηση της \en{dApp} είναι η ακόλουθη:


\begin{otherlanguage}{english}

\begin{lstlisting}[language=Java]

import java.math.BigInteger;
class Code {
  private String ver_string = "So Long, and Thanks for All the Fish";

  public String getVerification()
  {
    return ver_string;
  }

  public String getComputation()
  {
    BigInteger ret = BigInteger.ZERO;
    for (int i=0;i<100000;i++)
      for(int j=0;j<10000;j++)
       ret = ret.add(this.fibonacci(BigInteger.valueOf(42))); 
       //calculates the fibonacci value again and again for testing purposes
    return String.valueOf(ret);
  }

  private BigInteger fibonacci(BigInteger n) {
        if (n.equals(BigInteger.ZERO)) return BigInteger.ZERO;
        if (n.equals(BigInteger.ONE)) return BigInteger.ONE;

        BigInteger n1 = BigInteger.ZERO;
        BigInteger n2 = BigInteger.ONE;
        BigInteger n3 = BigInteger.ZERO;

        // Create a BigInteger for the loop counter i
        BigInteger i = BigInteger.valueOf(2);

        // Use the compareTo method to compare i with n
        while (i.compareTo(n) < 0) {
            n3 = n1.add(n2);
            n1 = n2;
            n2 = n3;
            
            // Increment i using the add method
            i = i.add(BigInteger.ONE);
        }
        return n3;
    }
}
\end{lstlisting}
\end{otherlanguage}



Σημειώνεται ότι η συνάρτηση \textit{\en{getComputation}} σκοπό είχε να είναι τόσο χρονοβόρα ώστε να προσφέρει τρόπο σύγκρισης του χρόνου εκτέλεσης της εργασίας και όχι να είναι αποδοτική.
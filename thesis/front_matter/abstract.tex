\begin{abstract}
Τα αποκεντρωμένα συστήματα διεισδύουν σταδιακά σε διάφορες πτυχές της ψηφιακής μας ζωής, προσφέροντας ενισχυμένο έλεγχο, ασφάλεια, διαφάνεια και προσβασιμότητα. Μια από τις πιο επιδραστικές εφαρμογές πάνω στις αρχές αυτές αποτελεί το υπολογιστικό νέφος, το οποίο μπορεί να εκδημοκρατίσει την πρόσβαση στους υπολογιστικούς πόρους. Η παρούσα διπλωματική εργασία, "Αποκεντρωμένο υπολογιστικό νέφος: Μια προσέγγιση βασισμένη στο \tl{Blockchain}", εμβαθύνει στις δυνατότητες αυτές εξετάζοντας την ανάπτυξη μιας αποκεντρωμένης εφαρμογής \tl{dApp} που ενσωματώνει τις αρχές της αποκέντρωσης στο πεδίο του υπολογιστικού νέφους.

Η \tl{dApp} που αναπτύχθηκε στο πλαίσιο της εργασίας, εισάγει ένα πλαίσιο για τον εντοπισμό, την εκτέλεση και την επαλήθευση υπολογιστικών εργασιών εντός ενός αποκεντρωμένου περιβάλλοντος. Αξιοποιεί την τεχνολογία \tl{Blockchain} και πιο συγκεκριμένα του \tl{Ethereum}, για διαφανείς και ασφαλείς συναλλαγές, έναν μηχανισμό δημοπρασίας για την δίκαιη κατανομή των υπολογιστικών εργασιών και το διαπλανητικο σύστημα αρχείων \tl{(IPFS)}, για την αμετάβλητη αποθήκευσή τους. Η εκτέλεση των εργασιών πραγματοποιείται μέσω του \tl{Docker}, παρέχοντας ένα ασφαλές και τυποποιημένο υπολογιστικό περιβάλλον ανάμεσα σε ένα δίκτυο μεμονωμένων παρόχων.

Στην συνέχεια, η μελέτη επιδίδεται περαιτέρω σε μια κριτική ανάλυση της \tl{dApp}, συγκρίνοντάς την με τις παραδοσιακές υπηρεσίες υπολογιστικού νέφους, για να αναδείξει τις δυνατότητές της για μεγαλύτερη διαφάνεια και εκδημοκρατισμό της πρόσβασης σε υπολογιστικούς πόρους. Ωστόσο, αναγνωρίζει τις υφιστάμενες προκλήσεις, κυρίως όσον αφορά την επεκτασιμότητα, την προσβασιμότητα των χρηστών και την αποδοτικότητά της ως προς το κόστος. 

Η εργασία ολοκληρώνεται με την προβολή του μελλοντικού τοπίου του αποκεντρωμένου υπολογιστικού νέφους, τονίζοντας την ανάγκη για συνεχή έρευνα σχετικά με την επεκτασιμότητα, τη διερεύνηση τεχνολογιών \tl{Blockchain} με χαμηλότερο λειτουργικό κόστος και τη συνεχή εξέλιξη των μηχανισμών ασφαλείας. Τέλος, μελεταται η δυνατότητα του \tl{Blockchain} να μετασχηματίσει το υπολογιστικό νέφος, υποστηρίζοντας τη στροφή προς ένα πιο δίκαιο ψηφιακο μέλλον.
   \begin{keywords}
      Αποκέντρωση, Υπολογιστικό Νέφος, \tl{Blockchain}, \tl{Ethereum}, Αποκεντρωμένες Εφαρμογές \tl{(dApps)}, Επεκτασιμότητα, Ασφάλεια, Διαφάνεια, Υπολογιστικές Εργασίες.
   \end{keywords}
\end{abstract}



\begin{abstracteng}
\tl{Decentralization is progressively permeating various aspects of our digital lives, promising enhanced control, security, and accessibility. One of the most impactful applications of these principles is in cloud computing, where decentralization stands to democratize access to computational resources. This thesis, "Decentralized Cloud Computing: A Blockchain-based Approach", delves into this potential by examining a specifically designed decentralized application (dApp) that embodies these principles in the realm of cloud computing.}

\tl{The dApp, developed as part of this research, introduces a framework for task allocation, execution, and verification within a decentralized environment. It leverages blockchain technology and more specifically Ethereum, for transparent and secure transactions, an auction mechanism for fair computational task allocation, and the InterPlanetary File System (IPFS) for their immutable storage. The execution of tasks is facilitated through Docker, providing a secure and standardized computing environment across a network of individual providers.}

\tl{This study further engages in a critical analysis of the dApp, placing it side by side with traditional cloud services to highlight its potential for greater transparency, and democratization of access to computing resources. However, it also acknowledges existing challenges, particularly concerning scalability, user accessibility and cost efficiency.}

\tl{The thesis concludes by projecting the future landscape of decentralized cloud computing, emphasizing the necessity for ongoing research into scalability, the exploration of blockchains with lower operational costs, and the continual evolution of security measures. This work serves as a study for understanding the transformative potential of blockchain in cloud computing, advocating for a shift towards a more equitable digital future.}



   \begin{keywordseng}
      \tl{Decentralization}, \tl{Cloud Computing}, \tl{Blockchain}, \tl{Ethereum}, \tl{Decentralized Application (dApps)}, \tl{Scalability}, \tl{Security}, \tl{Transparency}, \tl{Computational Task}.
   \end{keywordseng}

\end{abstracteng}